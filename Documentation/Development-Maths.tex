% !TeX spellcheck = en_US
% !TeX root = DynELA.tex
%
% LaTeX source file of DynELA FEM Code
%
% (c) by Olivier Pantalé 2020
%
\chapter{DynELA Maths library}

\startcontents[chapters]
\printmyminitoc[2]\LETTRINE{T}he \DynELA~is an Explicit FEM code written in \Cpp~using a Python's interface for creating the Finite Element Models. 

\section{The Tensor2 library}

The Tensor2 library is used to store second order tensors defined in the \DynELA.

A second order tensor is a like a matrix with the following form:
\begin{equation}
T=\left[\begin{array}{ccc}
  T_{11} & T_{12} & T_{13}\\
  T_{21} & T_{22} & T_{23}\\
  T_{31} & T_{32} & T_{33}
  \end{array}\right]
\end{equation}
Concerning the internal storage of data, the Tensor2 data is stored in a vector of 9 components using the following storage scheme:
\begin{equation}
T=\left[\begin{array}{ccc}
    T_{0} & T_{1} & T_{2}\\
    T_{3} & T_{4} & T_{5}\\
    T_{6} & T_{7} & T_{8}
    \end{array}\right]
\end{equation}

\subsection{Initialization and basic operations}
%@DOC:Tensor2::Tensor2()
%Warning :
%This area is an automatic documentation generated from the DynELA source code.
%Do not change anything in this latex file between this position and the @END keyword.
\textcolor{purple}{\textbf{Tensor2::Tensor2(~)}}\label{Tensor2::Tensor2()} : Default constructor of the Tensor2 class.\index[DL]{Tensor2!Tensor2(~)}\\ \hspace*{5mm}$\hookrightarrow$ Tensor2

All components are initialized to zero by default.
\begin{equation*}
\T=\left[\begin{array}{ccc}
0&0&0\\
0&0&0\\
0&0&0
\end{array}\right]
\end{equation*}
%@END

%@DOC:Tensor2::Tensor2(Tensor2)
%Warning :
%This area is an automatic documentation generated from the DynELA source code.
%Do not change anything in this latex file between this position and the @END keyword.
\textcolor{purple}{\textbf{Tensor2::Tensor2(Tensor2)}}\label{Tensor2::Tensor2(Tensor2)} : Copy constructor of the Tensor2 class.\index[DL]{Tensor2!Tensor2(Tensor2)}\\ \hspace*{5mm}$\hookrightarrow$ Tensor2

%@END

%@DOC:Tensor2::~Tensor2()
%Warning :
%This area is an automatic documentation generated from the DynELA source code.
%Do not change anything in this latex file between this position and the @END keyword.
\textcolor{purple}{\textbf{Tensor2::$\sim$Tensor2(~)}}\label{Tensor2::~Tensor2()} : Destructor of the Tensor2 class.\index[DL]{Tensor2!$\sim$Tensor2(~)}

%@END

%@DOC:Tensor2::Tensor2(double,...)
%Warning :
%This area is an automatic documentation generated from the DynELA source code.
%Do not change anything in this latex file between this position and the @END keyword.
\textcolor{purple}{\textbf{Tensor2::Tensor2(double,...)}}\label{Tensor2::Tensor2(double,...)} : Constructor of the Tensor2 class.\index[DL]{Tensor2!Tensor2(double,...)}\\ \hspace*{5mm}$\hookrightarrow$ Tensor2

Constructor of a second order tensor with initialization of the 9 values.
%@END

%@DOC:Tensor2::setToZero()
%Warning :
%This area is an automatic documentation generated from the DynELA source code.
%Do not change anything in this latex file between this position and the @END keyword.
\textcolor{purple}{\textbf{Tensor2::setToZero(~)}}\label{Tensor2::setToZero()} : Sets all components of the tensor to zero.\index[DL]{Tensor2!setToZero(~)}

\hspace*{10mm}\textcolor{red}{\textbf{Warning :} This method modifies its own argument}

\begin{equation*}
\T=\left[\begin{array}{ccc}
0&0&0\\
0&0&0\\
0&0&0
\end{array}\right]
\end{equation*}
%@END

%@DOC:Tensor2::setToUnity()
%Warning :
%This area is an automatic documentation generated from the DynELA source code.
%Do not change anything in this latex file between this position and the @END keyword.
\textcolor{purple}{\textbf{Tensor2::setToUnity(~)}}\label{Tensor2::setToUnity()} : Unity tensor.\index[DL]{Tensor2!setToUnity(~)}

\hspace*{10mm}\textcolor{red}{\textbf{Warning :} This method modifies its own argument}

This method transforms the current tensor to a unity tensor.
\begin{equation*}
\T=\left[\begin{array}{ccc}
1&0&0\\
0&1&0\\
0&0&1
\end{array}\right]
\end{equation*}
%@END

%@DOC:Tensor2::getTranspose()
%Warning :
%This area is an automatic documentation generated from the DynELA source code.
%Do not change anything in this latex file between this position and the @END keyword.
\textcolor{purple}{\textbf{Tensor2::getTranspose(~)}}\label{Tensor2::getTranspose()} : Transpose of a second order tensor.\index[DL]{Tensor2!getTranspose(~)}\\ \hspace*{5mm}$\hookrightarrow$ Tensor2

This method defines the transpose of a second second order tensor.
The result of this operation is a second order tensor defined by the following equation:
\begin{equation*}
\A=\B^T =\left[\begin{array}{ccc}
  B_{11} & B_{21} & B_{31}\\
  B_{12} & B_{22} & B_{32}\\
  B_{13} & B_{23} & B_{33}
  \end{array}\right]
\end{equation*}
%@END

%@DOC:Tensor2::operator=(double)
%Warning :
%This area is an automatic documentation generated from the DynELA source code.
%Do not change anything in this latex file between this position and the @END keyword.
\textcolor{purple}{\textbf{Tensor2::operator=(double)}}\label{Tensor2::operator=(double)} : Fill a second order tensor with a scalar value.\index[DL]{Tensor2!operator=(double)}\\ \hspace*{5mm}$\hookrightarrow$ Tensor2

This method is a surdefinition of the = operator for the second order tensor class.
\begin{equation*}
\T=\left[\begin{array}{ccc}
m&m&m\\
m&m&m\\
m&m&m
\end{array}\right]
\end{equation*}
%@END

%@DOC:Tensor2::rowSum()
%Warning :
%This area is an automatic documentation generated from the DynELA source code.
%Do not change anything in this latex file between this position and the @END keyword.
\textcolor{purple}{\textbf{Tensor2::rowSum(~)}}\label{Tensor2::rowSum()} : Sum of the rows of a second order tensor.\index[DL]{Tensor2!rowSum(~)}\\ \hspace*{5mm}$\hookrightarrow$ Vec3D

This method returns a vector by computing the sum of the components on all rows of a second second order tensor.
The result of this operation is a vector defined by:
\begin{equation*}
v_{i}=\sum_{j=1}^{3} T_{ji}
\end{equation*}
%@END

%@DOC:Tensor2::columnSum()
%Warning :
%This area is an automatic documentation generated from the DynELA source code.
%Do not change anything in this latex file between this position and the @END keyword.
\textcolor{purple}{\textbf{Tensor2::columnSum(~)}}\label{Tensor2::columnSum()} : Sum of the columns of a second order tensor.\index[DL]{Tensor2!columnSum(~)}\\ \hspace*{5mm}$\hookrightarrow$ Vec3D

This method returns a vector by computing the sum of the components on all columns of a second second order tensor.
The result of this operation is a vector defined by:
\begin{equation*}
v_{i}=\sum_{j=1}^{3}T_{ij}
\end{equation*}
%@END

%@DOC:Tensor2::getRow(short)
%Warning :
%This area is an automatic documentation generated from the DynELA source code.
%Do not change anything in this latex file between this position and the @END keyword.
\textcolor{purple}{\textbf{Tensor2::getRow(short)}}\label{Tensor2::getRow(short)} : Extraction of a row from a second order tensor.\index[DL]{Tensor2!getRow(short)}\\ \hspace*{5mm}$\hookrightarrow$ Vec3D

This method returns a vector as part of a second second order tensor.
The result of this operation with the argument j is a vector defined by:
\begin{equation*}
v_{i} = T_{ij}
\end{equation*}
%@END

%@DOC:Tensor2::getColumn(short)
%Warning :
%This area is an automatic documentation generated from the DynELA source code.
%Do not change anything in this latex file between this position and the @END keyword.
\textcolor{purple}{\textbf{Tensor2::getColumn(short)}}\label{Tensor2::getColumn(short)} : Extraction of a column from a second order tensor.\index[DL]{Tensor2!getColumn(short)}\\ \hspace*{5mm}$\hookrightarrow$ Vec3D

This method returns a vector as part of a second second order tensor.
The result of this operation with the argument j is a vector defined by:
\begin{equation*}
v_{i} = T_{ji}
\end{equation*}
%@END

%@DOC:Tensor2::minValue()
%Warning :
%This area is an automatic documentation generated from the DynELA source code.
%Do not change anything in this latex file between this position and the @END keyword.
\textcolor{purple}{\textbf{Tensor2::minValue(~)}}\label{Tensor2::minValue()} : Minimum component in a second order tensor.\index[DL]{Tensor2!minValue(~)}\\ \hspace*{5mm}$\hookrightarrow$ double

This method returns the minimum component in a second second order tensor.
%@END

%@DOC:Tensor2::minAbsoluteValue()
%Warning :
%This area is an automatic documentation generated from the DynELA source code.
%Do not change anything in this latex file between this position and the @END keyword.
\textcolor{purple}{\textbf{Tensor2::minAbsoluteValue(~)}}\label{Tensor2::minAbsoluteValue()} : Minimum absolute component in a second order tensor.\index[DL]{Tensor2!minAbsoluteValue(~)}\\ \hspace*{5mm}$\hookrightarrow$ double

This method returns the minimum absolute component in a second second order tensor.
%@END

%@DOC:Tensor2::maxValue()
%Warning :
%This area is an automatic documentation generated from the DynELA source code.
%Do not change anything in this latex file between this position and the @END keyword.
\textcolor{purple}{\textbf{Tensor2::maxValue(~)}}\label{Tensor2::maxValue()} : Maximum component in a second order tensor.\index[DL]{Tensor2!maxValue(~)}\\ \hspace*{5mm}$\hookrightarrow$ double

This method returns the maximum component in a second second order tensor.
%@END

%@DOC:Tensor2::maxAbsoluteValue()
%Warning :
%This area is an automatic documentation generated from the DynELA source code.
%Do not change anything in this latex file between this position and the @END keyword.
\textcolor{purple}{\textbf{Tensor2::maxAbsoluteValue(~)}}\label{Tensor2::maxAbsoluteValue()} : Maximum absolute component in a second order tensor.\index[DL]{Tensor2!maxAbsoluteValue(~)}\\ \hspace*{5mm}$\hookrightarrow$ double

This method returns the maximum absolute component in a second second order tensor.
%@END

%@DOC:Tensor2::getTrace()
%Warning :
%This area is an automatic documentation generated from the DynELA source code.
%Do not change anything in this latex file between this position and the @END keyword.
\textcolor{purple}{\textbf{Tensor2::getTrace(~)}}\label{Tensor2::getTrace()} : Returns the trace of a second order tensor.\index[DL]{Tensor2!getTrace(~)}\\ \hspace*{5mm}$\hookrightarrow$ double

  This method returns the trace of a second order tensor, i.e. the sum of all the terms of the diagonal:
\begin{equation*}
v = tr[\T] = T_{11}+T_{22}+T_{33}
\end{equation*}
%@END

%@DOC:Tensor2::getThirdTrace()
%Warning :
%This area is an automatic documentation generated from the DynELA source code.
%Do not change anything in this latex file between this position and the @END keyword.
\textcolor{purple}{\textbf{Tensor2::getThirdTrace(~)}}\label{Tensor2::getThirdTrace()} : Returns the average value of the trace of a second order tensor.\index[DL]{Tensor2!getThirdTrace(~)}\\ \hspace*{5mm}$\hookrightarrow$ double

This method returns average value of the trace of a second order tensor, i.e. the sum of all the terms of the diagonal divided by 3:
\begin{equation*}
v = \frac{1}{3} tr[\T] =  \frac{1}{3} \left( T_{11}+T_{22}+T_{33} \right)
\end{equation*}
%@END

\subsection{Specific operations}

%@DOC:Tensor2::singleProduct(Tensor2)
%Warning :
%This area is an automatic documentation generated from the DynELA source code.
%Do not change anything in this latex file between this position and the @END keyword.
\textcolor{purple}{\textbf{Tensor2::singleProduct(Tensor2)}}\label{Tensor2::singleProduct(Tensor2)} : Contracted product of two second order tensors.\index[DL]{Tensor2!singleProduct(Tensor2)}\\ \hspace*{5mm}$\hookrightarrow$ Tensor2

This method defines a single contracted product of two second order tensors.
The result of this operation is also a second order tensor defined by:
\begin{equation*}
\T = \A \cdot \B
\end{equation*}
where $\A$ and $\B$ are two second order tensors.
%@END

%@DOC:Tensor2::singleProduct()
%Warning :
%This area is an automatic documentation generated from the DynELA source code.
%Do not change anything in this latex file between this position and the @END keyword.
\textcolor{purple}{\textbf{Tensor2::singleProduct(~)}}\label{Tensor2::singleProduct()} : Contracted product of a second order tensor by itself.\index[DL]{Tensor2!singleProduct(~)}\\ \hspace*{5mm}$\hookrightarrow$ Tensor2

This method defines a single contracted product of of a second order tensor by itself.
The result of this operation is also a second order tensor defined by:
\begin{equation*}
\T = \A \cdot \A
\end{equation*}
where $\A$ is a two second order tensor.
%@END

%@DOC:Tensor2::singleProductTxN()
%Warning :
%This area is an automatic documentation generated from the DynELA source code.
%Do not change anything in this latex file between this position and the @END keyword.
\textcolor{purple}{\textbf{Tensor2::singleProductTxN(~)}}\label{Tensor2::singleProductTxN()} : Contracted product of a second order tensor by its transpose.\index[DL]{Tensor2!singleProductTxN(~)}\\ \hspace*{5mm}$\hookrightarrow$ SymTensor2

This method defines a single contracted product of two second order tensors.
The result of this operation is also a second order tensor defined by:
\begin{equation*}
\T = \A^T\cdot \A
\end{equation*}
where $\A$ is a second order tensor. Result is a symmetric second order tensor.
%@END

%@DOC:Tensor2::singleProductNxT()
%Warning :
%This area is an automatic documentation generated from the DynELA source code.
%Do not change anything in this latex file between this position and the @END keyword.
\textcolor{purple}{\textbf{Tensor2::singleProductNxT(~)}}\label{Tensor2::singleProductNxT()} : Contracted product of a second order tensor by its transpose.\index[DL]{Tensor2!singleProductNxT(~)}\\ \hspace*{5mm}$\hookrightarrow$ SymTensor2

This method defines a single contracted product of two second order tensors.
The result of this operation is also a second order tensor defined by:
\begin{equation*}
\T = \A \cdot \A^T
\end{equation*}
where $\A$ is a second order tensor. Result is a symmetric second order tensor.
%@END

%@DOC:Tensor2::operator*(Tensor2)
%Warning :
%This area is an automatic documentation generated from the DynELA source code.
%Do not change anything in this latex file between this position and the @END keyword.
\textcolor{purple}{\textbf{Tensor2::operator*(Tensor2)}}\label{Tensor2::operator*(Tensor2)} : Multiplication of 2 second order tensors.\index[DL]{Tensor2!operator*(Tensor2)}\\ \hspace*{5mm}$\hookrightarrow$ Tensor2

This method defines a single contracted product of two second order tensors.
The result of this operation is also a second order tensor defined by:
\begin{equation*}
\T = \A \cdot \B
\end{equation*}
where $\A$ and $\B$ are two second order tensors.
%@END

%@DOC:Tensor2::operator*(SymTensor2)
%Warning :
%This area is an automatic documentation generated from the DynELA source code.
%Do not change anything in this latex file between this position and the @END keyword.
\textcolor{purple}{\textbf{Tensor2::operator*(SymTensor2)}}\label{Tensor2::operator*(SymTensor2)} : Multiplication of 2 second order tensors.\index[DL]{Tensor2!operator*(SymTensor2)}\\ \hspace*{5mm}$\hookrightarrow$ Tensor2

This method defines a single contracted product of two second order tensors.
The result of this operation is also a second order tensor defined by:
\begin{equation*}
\T = \A \cdot \B
\end{equation*}
where $\A$ is a second order tensor and $\B$ is a symmetric second order tensor.
%@END

%@DOC:Tensor2::doubleProduct(Tensor2)
%Warning :
%This area is an automatic documentation generated from the DynELA source code.
%Do not change anything in this latex file between this position and the @END keyword.
\textcolor{purple}{\textbf{Tensor2::doubleProduct(Tensor2)}}\label{Tensor2::doubleProduct(Tensor2)} : Double contracted product of 2 second order tensors.\index[DL]{Tensor2!doubleProduct(Tensor2)}\\ \hspace*{5mm}$\hookrightarrow$ double

This method defines a double contracted product of two second order tensors.
The result of this operation is a scalar defined by:
\begin{equation*}
s = \A : \B = \sum_{i=1}^{3} \sum_{j=1}^{3} A_{ij}\times B_{ij}
\end{equation*}
where $\A$ and $\B$ are two second order tensors.
%@END

%@DOC:Tensor2::doubleProduct()
%Warning :
%This area is an automatic documentation generated from the DynELA source code.
%Do not change anything in this latex file between this position and the @END keyword.
\textcolor{purple}{\textbf{Tensor2::doubleProduct(~)}}\label{Tensor2::doubleProduct()} : Double contracted product of a second order tensor by itself.\index[DL]{Tensor2!doubleProduct(~)}\\ \hspace*{5mm}$\hookrightarrow$ double

This method defines a double contracted product of a second order tensor by itself.
The result of this operation is a scalar defined by:
\begin{equation*}
s = \A : \A = \sum_{i=1}^{3} \sum_{j=1}^{3} A_{ij}\times A_{ij}
\end{equation*}
where $\A$ is a second order tensor.
%@END

%@DOC:Tensor2::operator*(Vec3D)
%Warning :
%This area is an automatic documentation generated from the DynELA source code.
%Do not change anything in this latex file between this position and the @END keyword.
\textcolor{purple}{\textbf{Tensor2::operator*(Vec3D)}}\label{Tensor2::operator*(Vec3D)} : Multiplication of a second order tensor by a vector.\index[DL]{Tensor2!operator*(Vec3D)}\\ \hspace*{5mm}$\hookrightarrow$ Vec3D

  This method defines the product of a second order tensor by a vector.
  The result of this operation is also a vector defined by:
\begin{equation*}
\overrightarrow{y} = \A \cdot \overrightarrow{x}
\end{equation*}
where $\A$ is a second order tensor and $\overrightarrow{x}$ and $\overrightarrow{y}$ are two Vec3D.
%@END

%@DOC:Tensor2::getDeviator()
%Warning :
%This area is an automatic documentation generated from the DynELA source code.
%Do not change anything in this latex file between this position and the @END keyword.
\textcolor{purple}{\textbf{Tensor2::getDeviator(~)}}\label{Tensor2::getDeviator()} : Deviatoric part of a second order tensor.\index[DL]{Tensor2!getDeviator(~)}\\ \hspace*{5mm}$\hookrightarrow$ Tensor2

This method defines the deviatoric part of a second second order tensor.
The result of this operation is a second order tensor defined by the following equation:
\begin{equation*}
\Sig^d=\Sig-\frac{1}{3}\tr[\Sig].\Id
\end{equation*}
where $\Sig^d$ is the deviatoric part of the tensor, $\Sig$ is the tensor and $\Id$ is the unit tensor.
%@END

%@DOC:Tensor2::getSymetricPart()
%Warning :
%This area is an automatic documentation generated from the DynELA source code.
%Do not change anything in this latex file between this position and the @END keyword.
\textcolor{purple}{\textbf{Tensor2::getSymetricPart(~)}}\label{Tensor2::getSymetricPart()} : Symmetric part of a second order tensor.\index[DL]{Tensor2!getSymetricPart(~)}\\ \hspace*{5mm}$\hookrightarrow$ Tensor2

This method returns the symmetric part of a second second order tensor.
The result of this operation is a second second order tensor defined by:
\begin{equation*}
\B = \left[\begin{array}{ccc}
 A_{11} & \frac{A_{12} + A_{21}}{2} & \frac{A_{13} + A_{31}}{2}\\
 \frac{A_{12} + A_{21}}{2} & A_{22} & \frac {A_{23} + A_{32}}{2}\\
 \frac{A_{13} + A_{31}}{2} & \frac {A_{23} + A_{32}}{2} & A_{33}\end{array}
\right]
\end{equation*}
%@END

%@DOC:Tensor2::getSkewSymetricPart()
%Warning :
%This area is an automatic documentation generated from the DynELA source code.
%Do not change anything in this latex file between this position and the @END keyword.
\textcolor{purple}{\textbf{Tensor2::getSkewSymetricPart(~)}}\label{Tensor2::getSkewSymetricPart()} : Skew-symmetric part of a second order tensor.\index[DL]{Tensor2!getSkewSymetricPart(~)}\\ \hspace*{5mm}$\hookrightarrow$ Tensor2

This method returns the skew-symmetric part of a second second order tensor.
The result of this operation is a second second order tensor defined by:
\begin{equation*}
\B = \left[\begin{array}{ccc}
 A_{11} & \frac{A_{12} - A_{21}}{2} & \frac{A_{13} - A_{31}}{2}\\
 -\frac{A_{12} -  A_{21}}{2} & A_{22} & \frac {A_{23} - A_{32}}{2}\\
 -\frac{A_{13} - A_{31}}{2} & -\frac {A_{23} - A_{32}}{2} & A_{33}\end{array}
\right]
\end{equation*}
%@END

%@DOC:Tensor2::getDeterminant()
%Warning :
%This area is an automatic documentation generated from the DynELA source code.
%Do not change anything in this latex file between this position and the @END keyword.
\textcolor{purple}{\textbf{Tensor2::getDeterminant(~)}}\label{Tensor2::getDeterminant()} : Determinant of a second order tensor.\index[DL]{Tensor2!getDeterminant(~)}\\ \hspace*{5mm}$\hookrightarrow$ double

This method returns the determinant of a second second order tensor.
The result of this operation is a scalar value defined by:
\begin{equation*}
D = T_{11} T_{22} T_{33} + T_{21} T_{32} T_{13} + T_{31} T_{12} T_{23} - T_{31} T_{22} T_{13} - T_{11} T_{32} T_{23} - T_{21} T_{12} T_{33}
\end{equation*}
%@END

%@DOC:Tensor2::getInverse()
%Warning :
%This area is an automatic documentation generated from the DynELA source code.
%Do not change anything in this latex file between this position and the @END keyword.
\textcolor{purple}{\textbf{Tensor2::getInverse(~)}}\label{Tensor2::getInverse()} : Inverse of a second order tensor.\index[DL]{Tensor2!getInverse(~)}\\ \hspace*{5mm}$\hookrightarrow$ Tensor2

This method returns the inverse of a second second order tensor.
The result of this operation is a second order tensor defined by:
\begin{equation*}
D = T_{11} T_{22} T_{33} + T_{21} T_{32} T_{13} + T_{31} T_{12} T_{23} - T_{31} T_{22} T_{13} - T_{11} T_{32} T_{23} - T_{21} T_{12} T_{33}
\end{equation*}
\begin{equation*}
T^{-1} = \frac {1}{D} \left[\begin{array}{ccc}
  T_{22}T_{33}-T_{23}T_{32}&T_{13}T_{32}-T_{12}T_{33}&T_{12}T_{23}-T_{13}T_{22}\\
  T_{23}T_{31}-T_{21}T_{33}&T_{11}T_{33}-T_{13}T_{31}&T_{13}T_{21}-T_{11}T_{23}\\
  T_{21}T_{32}-T_{22}T_{31}&T_{12}T_{31}-T_{11}T_{32}&T_{11}T_{22}-T_{12}T_{21}
  \end{array}
  \right]
\end{equation*}
%@END

%@DOC:Tensor2::getNorm()
%Warning :
%This area is an automatic documentation generated from the DynELA source code.
%Do not change anything in this latex file between this position and the @END keyword.
\textcolor{purple}{\textbf{Tensor2::getNorm(~)}}\label{Tensor2::getNorm()} : Norm of a second order tensor.\index[DL]{Tensor2!getNorm(~)}\\ \hspace*{5mm}$\hookrightarrow$ double

This method returns the norm of a second order tensor defined by:\begin{equation*}
\left\Vert s \right\Vert  = \sqrt {s_{ij}:s_{ij}}
\end{equation*}
%@END

%@DOC:Tensor2::getJ2()
%Warning :
%This area is an automatic documentation generated from the DynELA source code.
%Do not change anything in this latex file between this position and the @END keyword.
\textcolor{purple}{\textbf{Tensor2::getJ2(~)}}\label{Tensor2::getJ2()} : J2 norm of a second order tensor.\index[DL]{Tensor2!getJ2(~)}\\ \hspace*{5mm}$\hookrightarrow$ double

This method returns the J2 norm of a second order tensor defined by:
\begin{equation*}
\sqrt {\frac{3}{2}} \left\Vert s \right\Vert  = \sqrt {\frac{3}{2} s_{ij}:s_{ij}}
\end{equation*}
%@END

\section{The Symmetric Tensor2 library}

The SymTensor2 library is used to store symmetric second order tensors defined in the \DynELA.

A symmetric second order tensor is a like a matrix with the following form:
\begin{equation}
T=\left[\begin{array}{ccc}
  T_{11} & T_{12} & T_{13}\\
  T_{12} & T_{22} & T_{23}\\
  T_{13} & T_{23} & T_{33}
  \end{array}\right]
\end{equation}
Concerning the internal storage of data, the SymTensor2 data is stored in a vector of 9 components using the following storage scheme:
\begin{equation}
T=\left[\begin{array}{ccc}
    T_{0} & T_{1} & T_{2}\\
    T_{1} & T_{3} & T_{4}\\
    T_{2} & T_{4} & T_{5}
    \end{array}\right]
\end{equation}
\subsection{Initialization and basic operations}
%@DOC:SymTensor2::SymTensor2()
%Warning :
%This area is an automatic documentation generated from the DynELA source code.
%Do not change anything in this latex file between this position and the @END keyword.
\textcolor{purple}{\textbf{SymTensor2::SymTensor2(~)}}\label{SymTensor2::SymTensor2()} : Default constructor of the SymTensor2 class.\index[DL]{SymTensor2!SymTensor2(~)}\\ \hspace*{5mm}$\hookrightarrow$ SymTensor2

All components are initialized to zero by default.
\begin{equation*}
\T=\left[\begin{array}{ccc}
0&0&0\\
0&0&0\\
0&0&0
\end{array}\right]
\end{equation*}
%@END

%@DOC:SymTensor2::SymTensor2(SymTensor2)
%Warning :
%This area is an automatic documentation generated from the DynELA source code.
%Do not change anything in this latex file between this position and the @END keyword.
\textcolor{purple}{\textbf{SymTensor2::SymTensor2(SymTensor2)}}\label{SymTensor2::SymTensor2(SymTensor2)} : Copy constructor of the SymTensor2 class.\index[DL]{SymTensor2!SymTensor2(SymTensor2)}\\ \hspace*{5mm}$\hookrightarrow$ SymTensor2

%@END

%@DOC:SymTensor2::~SymTensor2()
%Warning :
%This area is an automatic documentation generated from the DynELA source code.
%Do not change anything in this latex file between this position and the @END keyword.
\textcolor{purple}{\textbf{SymTensor2::$\sim$SymTensor2(~)}}\label{SymTensor2::~SymTensor2()} : Destructor of the SymTensor2 class.\index[DL]{SymTensor2!$\sim$SymTensor2(~)}

%@END

%@DOC:SymTensor2::SymTensor2(double,...)
%Warning :
%This area is an automatic documentation generated from the DynELA source code.
%Do not change anything in this latex file between this position and the @END keyword.
\textcolor{purple}{\textbf{SymTensor2::SymTensor2(double,...)}}\label{SymTensor2::SymTensor2(double,...)} : Constructor of the SymTensor2 class.\index[DL]{SymTensor2!SymTensor2(double,...)}\\ \hspace*{5mm}$\hookrightarrow$ SymTensor2

Constructor of a second order tensor with initialization of the 9 values.
%@END

%@DOC:SymTensor2::setToZero()
%Warning :
%This area is an automatic documentation generated from the DynELA source code.
%Do not change anything in this latex file between this position and the @END keyword.
\textcolor{purple}{\textbf{SymTensor2::setToZero(~)}}\label{SymTensor2::setToZero()} : Sets all components of the tensor to zero.\index[DL]{SymTensor2!setToZero(~)}

\hspace*{10mm}\textcolor{red}{\textbf{Warning :} This method modifies its own argument}

\begin{equation*}
\T=\left[\begin{array}{ccc}
0&0&0\\
0&0&0\\
0&0&0
\end{array}\right]
\end{equation*}
%@END

%@DOC:SymTensor2::setToUnity()
%Warning :
%This area is an automatic documentation generated from the DynELA source code.
%Do not change anything in this latex file between this position and the @END keyword.
\textcolor{purple}{\textbf{SymTensor2::setToUnity(~)}}\label{SymTensor2::setToUnity()} : Unity tensor.\index[DL]{SymTensor2!setToUnity(~)}

\hspace*{10mm}\textcolor{red}{\textbf{Warning :} This method modifies its own argument}

This method transforms the current tensor to a unity tensor.
\begin{equation*}
\T=\left[\begin{array}{ccc}
1&0&0\\
0&1&0\\
0&0&1
\end{array}\right]
\end{equation*}
%@END

%@DOC:SymTensor2::operator=(double)
%Warning :
%This area is an automatic documentation generated from the DynELA source code.
%Do not change anything in this latex file between this position and the @END keyword.
\textcolor{purple}{\textbf{SymTensor2::operator=(double)}}\label{SymTensor2::operator=(double)} : Fill a second order tensor with a scalar value.\index[DL]{SymTensor2!operator=(double)}\\ \hspace*{5mm}$\hookrightarrow$ SymTensor2

This method is a surdefinition of the = operator for the symmetric second order tensor class.
\begin{equation*}
\T=\left[\begin{array}{ccc}
m&m&m\\
m&m&m\\
m&m&m
\end{array}\right]
\end{equation*}
%@END

%@DOC:SymTensor2::rowSum()
%Warning :
%This area is an automatic documentation generated from the DynELA source code.
%Do not change anything in this latex file between this position and the @END keyword.
\textcolor{purple}{\textbf{SymTensor2::rowSum(~)}}\label{SymTensor2::rowSum()} : Sum of the rows of a second order tensor.\index[DL]{SymTensor2!rowSum(~)}\\ \hspace*{5mm}$\hookrightarrow$ Vec3D

This method returns a vector by computing the sum of the components on all rows of a second second order tensor.
The result of this operation is a vector defined by:
\begin{equation*}
v_{i}=\sum_{j=1}^{3} T_{ji}
\end{equation*}
%@END

%@DOC:SymTensor2::columnSum()
%Warning :
%This area is an automatic documentation generated from the DynELA source code.
%Do not change anything in this latex file between this position and the @END keyword.
\textcolor{purple}{\textbf{SymTensor2::columnSum(~)}}\label{SymTensor2::columnSum()} : Sum of the columns of a second order tensor.\index[DL]{SymTensor2!columnSum(~)}\\ \hspace*{5mm}$\hookrightarrow$ Vec3D

This method returns a vector by computing the sum of the components on all columns of a second second order tensor.
The result of this operation is a vector defined by:
\begin{equation*}
v_{i}=\sum_{j=1}^{3}T_{ij}
\end{equation*}
%@END

%@DOC:SymTensor2::getRow(short)
%Warning :
%This area is an automatic documentation generated from the DynELA source code.
%Do not change anything in this latex file between this position and the @END keyword.
\textcolor{purple}{\textbf{SymTensor2::getRow(short)}}\label{SymTensor2::getRow(short)} : Extraction of a row from a second order tensor.\index[DL]{SymTensor2!getRow(short)}\\ \hspace*{5mm}$\hookrightarrow$ Vec3D

This method returns a vector as part of a second second order tensor.
The result of this operation with the argument j is a vector defined by:
\begin{equation*}
v_{i} = T_{ij}
\end{equation*}
%@END

%@DOC:SymTensor2::getColumn(short)
%Warning :
%This area is an automatic documentation generated from the DynELA source code.
%Do not change anything in this latex file between this position and the @END keyword.
\textcolor{purple}{\textbf{SymTensor2::getColumn(short)}}\label{SymTensor2::getColumn(short)} : Extraction of a column from a second order tensor.\index[DL]{SymTensor2!getColumn(short)}\\ \hspace*{5mm}$\hookrightarrow$ Vec3D

This method returns a vector as part of a second second order tensor.
The result of this operation with the argument j is a vector defined by:
\begin{equation*}
v_{i} = T_{ji}
\end{equation*}
%@END

%@DOC:SymTensor2::minValue()
%Warning :
%This area is an automatic documentation generated from the DynELA source code.
%Do not change anything in this latex file between this position and the @END keyword.
\textcolor{purple}{\textbf{SymTensor2::minValue(~)}}\label{SymTensor2::minValue()} : Minimum component in a second order tensor.\index[DL]{SymTensor2!minValue(~)}\\ \hspace*{5mm}$\hookrightarrow$ double

This method returns the minimum component in a second second order tensor.
%@END

%@DOC:SymTensor2::minAbsoluteValue()
%Warning :
%This area is an automatic documentation generated from the DynELA source code.
%Do not change anything in this latex file between this position and the @END keyword.
\textcolor{purple}{\textbf{SymTensor2::minAbsoluteValue(~)}}\label{SymTensor2::minAbsoluteValue()} : Minimum absolute component in a second order tensor.\index[DL]{SymTensor2!minAbsoluteValue(~)}\\ \hspace*{5mm}$\hookrightarrow$ double

This method returns the minimum absolute component in a second second order tensor.
%@END

%@DOC:SymTensor2::maxValue()
%Warning :
%This area is an automatic documentation generated from the DynELA source code.
%Do not change anything in this latex file between this position and the @END keyword.
\textcolor{purple}{\textbf{SymTensor2::maxValue(~)}}\label{SymTensor2::maxValue()} : Maximum component in a second order tensor.\index[DL]{SymTensor2!maxValue(~)}\\ \hspace*{5mm}$\hookrightarrow$ double

This method returns the maximum component in a second second order tensor.
%@END

%@DOC:SymTensor2::maxAbsoluteValue()
%Warning :
%This area is an automatic documentation generated from the DynELA source code.
%Do not change anything in this latex file between this position and the @END keyword.
\textcolor{purple}{\textbf{SymTensor2::maxAbsoluteValue(~)}}\label{SymTensor2::maxAbsoluteValue()} : Maximum absolute component in a second order tensor.\index[DL]{SymTensor2!maxAbsoluteValue(~)}\\ \hspace*{5mm}$\hookrightarrow$ double

This method returns the maximum absolute component in a second second order tensor.
%@END

%@DOC:SymTensor2::getTrace()
%Warning :
%This area is an automatic documentation generated from the DynELA source code.
%Do not change anything in this latex file between this position and the @END keyword.
\textcolor{purple}{\textbf{SymTensor2::getTrace(~)}}\label{SymTensor2::getTrace()} : Returns the trace of a symmetric second order tensor.\index[DL]{SymTensor2!getTrace(~)}\\ \hspace*{5mm}$\hookrightarrow$ double

  This method returns the trace of a symmetric second order tensor, i.e. the sum of all the terms of the diagonal:
\begin{equation*}
v = tr[\T] = T_{11}+T_{22}+T_{33}
\end{equation*}
%@END

%@DOC:SymTensor2::getThirdTrace()
%Warning :
%This area is an automatic documentation generated from the DynELA source code.
%Do not change anything in this latex file between this position and the @END keyword.
\textcolor{purple}{\textbf{SymTensor2::getThirdTrace(~)}}\label{SymTensor2::getThirdTrace()} : Returns the average value of the trace of a symmetric second order tensor.\index[DL]{SymTensor2!getThirdTrace(~)}\\ \hspace*{5mm}$\hookrightarrow$ double

This method returns average value of the trace of a symmetric second order tensor, i.e. the sum of all the terms of the diagonal divided by 3:
\begin{equation*}
v = \frac{1}{3} tr[\T] =  \frac{1}{3} \left( T_{11}+T_{22}+T_{33} \right)
\end{equation*}
%@END

\subsection{Specific operations}

%@DOC:SymTensor2::singleProduct(SymTensor2)
%Warning :
%This area is an automatic documentation generated from the DynELA source code.
%Do not change anything in this latex file between this position and the @END keyword.
\textcolor{purple}{\textbf{SymTensor2::singleProduct(SymTensor2)}}\label{SymTensor2::singleProduct(SymTensor2)} : Contracted product of two symmetric second order tensors.\index[DL]{SymTensor2!singleProduct(SymTensor2)}\\ \hspace*{5mm}$\hookrightarrow$ Tensor2

This method defines a single contracted product of two symmetric second order tensors.
The result of this operation is also a second order tensor defined by:
\begin{equation*}
\T=A \cdot \B=\left[\begin{array}{ccc}
A_{11} B_{11} + A_{12} B_{12} + A_{13} B_{13} & A_{11} B_{12} + A_{12} B_{22} + A_{13} B_{23} & A_{11} B_{13} + A_{12} B_{23} + A_{13} B_{33} \\
A_{12} B_{11} + A_{22} B_{12} + A_{23} B_{13} & A_{12} B_{12} + A_{22} B_{22} + A_{23} B_{23} & A_{12} B_{13} + A_{22} B_{23} + A_{23} B_{33} \\
A_{13} B_{11} + A_{23} B_{12} + A_{33} B_{13} & A_{13} B_{12} + A_{23} B_{22} + A_{33} B_{23} & A_{13} B_{13} + A_{23} B_{23} + A_{33} B_{33}
\end{array}\right]
\end{equation*}
where $\A$ and $\B$ are two symmetric second order tensors, the result is a non symmetric second order tensor.
%@END

%@DOC:SymTensor2::singleProduct()
%Warning :
%This area is an automatic documentation generated from the DynELA source code.
%Do not change anything in this latex file between this position and the @END keyword.
\textcolor{purple}{\textbf{SymTensor2::singleProduct(~)}}\label{SymTensor2::singleProduct()} : Contracted product of a symmetric second order tensor by itself.\index[DL]{SymTensor2!singleProduct(~)}\\ \hspace*{5mm}$\hookrightarrow$ SymTensor2

This method defines a single contracted product of of a symmetric second order tensor by itself.
The result of this operation is also a symmetric second order tensor defined by:
\begin{equation*}
\T = \A \cdot \A
\end{equation*}
where $\A$ is a two symmetric second order tensor.
%@END

%@DOC:SymTensor2::operator*(SymTensor2)
%Warning :
%This area is an automatic documentation generated from the DynELA source code.
%Do not change anything in this latex file between this position and the @END keyword.
\textcolor{purple}{\textbf{SymTensor2::operator*(SymTensor2)}}\label{SymTensor2::operator*(SymTensor2)} : Multiplication of 2 second order tensors.\index[DL]{SymTensor2!operator*(SymTensor2)}\\ \hspace*{5mm}$\hookrightarrow$ Tensor2

This method defines a single contracted product of two symmetric second order tensors.
The result of this operation is also a second order tensor defined by:
\begin{equation*}
\T = \A \cdot \B
\end{equation*}
where $\A$ and $\B$ are two symmetric second order tensors.
%@END

%@DOC:SymTensor2::operator*(Tensor2)
%Warning :
%This area is an automatic documentation generated from the DynELA source code.
%Do not change anything in this latex file between this position and the @END keyword.
\textcolor{purple}{\textbf{SymTensor2::operator*(Tensor2)}}\label{SymTensor2::operator*(Tensor2)} : Multiplication of 2 second order tensors.\index[DL]{SymTensor2!operator*(Tensor2)}\\ \hspace*{5mm}$\hookrightarrow$ Tensor2

This method defines a single contracted product of two symmetric second order tensors.
The result of this operation is also a second order tensor defined by:
\begin{equation*}
\T = \A \cdot \B
\end{equation*}
where $\B$ is a second order tensor and $\A$ is a symmetric second order tensor.
%@END

%@DOC:SymTensor2::doubleProduct(SymTensor2)
%Warning :
%This area is an automatic documentation generated from the DynELA source code.
%Do not change anything in this latex file between this position and the @END keyword.
\textcolor{purple}{\textbf{SymTensor2::doubleProduct(SymTensor2)}}\label{SymTensor2::doubleProduct(SymTensor2)} : Double contracted product of 2 second order tensors.\index[DL]{SymTensor2!doubleProduct(SymTensor2)}\\ \hspace*{5mm}$\hookrightarrow$ double

This method defines a double contracted product of two symmetric second order tensors.
The result of this operation is a scalar defined by:
\begin{equation*}
s = \A : \B = \sum_{i=1}^{3} \sum_{j=1}^{3} A_{ij}\times B_{ij}
\end{equation*}
where $\A$ and $\B$ are two symmetric second order tensors.
%@END

%@DOC:SymTensor2::doubleProduct()
%Warning :
%This area is an automatic documentation generated from the DynELA source code.
%Do not change anything in this latex file between this position and the @END keyword.
\textcolor{purple}{\textbf{SymTensor2::doubleProduct(~)}}\label{SymTensor2::doubleProduct()} : Double contracted product of a second order tensor by itself.\index[DL]{SymTensor2!doubleProduct(~)}\\ \hspace*{5mm}$\hookrightarrow$ double

This method defines a double contracted product of a second order tensor by itself.
The result of this operation is a scalar defined by:
\begin{equation*}
s = \A : \A = \sum_{i=1}^{3} \sum_{j=1}^{3} A_{ij}\times A_{ij}
\end{equation*}
where $\A$ is a second order tensor.
%@END

%@DOC:SymTensor2::operator*(Vec3D)
%Warning :
%This area is an automatic documentation generated from the DynELA source code.
%Do not change anything in this latex file between this position and the @END keyword.
\textcolor{purple}{\textbf{SymTensor2::operator*(Vec3D)}}\label{SymTensor2::operator*(Vec3D)} : Multiplication of a symmetric second order tensor by a vector.\index[DL]{SymTensor2!operator*(Vec3D)}\\ \hspace*{5mm}$\hookrightarrow$ Vec3D

  This method defines the product of a symmetric second order tensor by a vector.
  The result of this operation is also a vector defined by:
\begin{equation*}
\overrightarrow{y} = \A \cdot \overrightarrow{x}
\end{equation*}
where $\A$ is a symmetric second order tensor and $\overrightarrow{x}$ and $\overrightarrow{y}$ are two Vec3D.
%@END

%@DOC:SymTensor2::getDeviator()
%Warning :
%This area is an automatic documentation generated from the DynELA source code.
%Do not change anything in this latex file between this position and the @END keyword.
\textcolor{purple}{\textbf{SymTensor2::getDeviator(~)}}\label{SymTensor2::getDeviator()} : Deviatoric part of a second order tensor.\index[DL]{SymTensor2!getDeviator(~)}\\ \hspace*{5mm}$\hookrightarrow$ SymTensor2

This method defines the deviatoric part of a second second order tensor.
The result of this operation is a second order tensor defined by the following equation:
\begin{equation*}
\Sig^d=\Sig-\frac{1}{3}\tr[\Sig].\Id
\end{equation*}
where $\Sig^d$ is the deviatoric part of the tensor, $\Sig$ is the tensor and $\Id$ is the unit tensor.
%@END

%@DOC:SymTensor2::getDeterminant()
%Warning :
%This area is an automatic documentation generated from the DynELA source code.
%Do not change anything in this latex file between this position and the @END keyword.
\textcolor{purple}{\textbf{SymTensor2::getDeterminant(~)}}\label{SymTensor2::getDeterminant()} : Determinant of a symmetric second order tensor.\index[DL]{SymTensor2!getDeterminant(~)}\\ \hspace*{5mm}$\hookrightarrow$ double

This method returns the determinant of a symmetric second second order tensor.
The result of this operation is a scalar value defined by:
\begin{equation*}
D = T_{11} T_{22} T_{33} + 2 T_{12} T_{23} T_{13} - T_{22} T_{13}^2 - T_{11} T_{23}^2 - T_{33} T_{12}^2
\end{equation*}
%@END

%@DOC:SymTensor2::getInverse()
%Warning :
%This area is an automatic documentation generated from the DynELA source code.
%Do not change anything in this latex file between this position and the @END keyword.
\textcolor{purple}{\textbf{SymTensor2::getInverse(~)}}\label{SymTensor2::getInverse()} : Inverse of a second order tensor.\index[DL]{SymTensor2!getInverse(~)}\\ \hspace*{5mm}$\hookrightarrow$ SymTensor2

This method returns the inverse of a second second order tensor.
The result of this operation is a second order tensor defined by:
\begin{equation*}
D = T_{11} T_{22} T_{33} + 2 T_{12} T_{23} T_{13} - T_{22} T_{13}^2 - T_{11} T_{23}^2 - T_{33} T_{12}^2
\end{equation*}
\begin{equation*}
T^{-1} = \frac {1}{D} \left[\begin{array}{ccc}
  T_{22}T_{33}-T_{23}^2&T_{13}T_{23}-T_{12}T_{33}&T_{12}T_{23}-T_{13}T_{22}\\
  T_{13}T_{23}-T_{12}T_{33}&T_{11}T_{33}-T_{13}^2&T_{12}T_{13}-T_{11}T_{23}\\
  T_{12}T_{23}-T_{13}T_{22}&T_{12}T_{13}-T_{11}T_{23}&T_{11}T_{22}-T_{12}^2
  \end{array}
  \right]
\end{equation*}
%@END

%@DOC:SymTensor2::getNorm()
%Warning :
%This area is an automatic documentation generated from the DynELA source code.
%Do not change anything in this latex file between this position and the @END keyword.
\textcolor{purple}{\textbf{SymTensor2::getNorm(~)}}\label{SymTensor2::getNorm()} : Norm of a symmetric second order tensor.\index[DL]{SymTensor2!getNorm(~)}\\ \hspace*{5mm}$\hookrightarrow$ double

This method returns the norm of a symmetric second order tensor defined by:
\begin{equation*}
\left\Vert s \right\Vert  = \sqrt {s_{ij}:s_{ij}}
\end{equation*}
%@END

%@DOC:SymTensor2::getJ2()
%Warning :
%This area is an automatic documentation generated from the DynELA source code.
%Do not change anything in this latex file between this position and the @END keyword.
\textcolor{purple}{\textbf{SymTensor2::getJ2(~)}}\label{SymTensor2::getJ2()} : J2 norm of a symmetric second order tensor.\index[DL]{SymTensor2!getJ2(~)}\\ \hspace*{5mm}$\hookrightarrow$ double

This method returns the J2 norm of a symmetric second order tensor defined by:
\begin{equation*}
\sqrt {\frac{3}{2}} \left\Vert s \right\Vert  = \sqrt {\frac{3}{2} s_{ij}:s_{ij}}
\end{equation*}
%@END

%@DOC:SymTensor2::getMisesEquivalent()
%Warning :
%This area is an automatic documentation generated from the DynELA source code.
%Do not change anything in this latex file between this position and the @END keyword.
\textcolor{purple}{\textbf{SymTensor2::getMisesEquivalent(~)}}\label{SymTensor2::getMisesEquivalent()} : Returns the von Mises stress of a symmetric second order tensor.\index[DL]{SymTensor2!getMisesEquivalent(~)}\\ \hspace*{5mm}$\hookrightarrow$ double

This method returns the von Mises stress of a symmetric second order tensor defined by:
\begin{equation*}
\overline{\sigma} = \frac {1}{\sqrt{2}}\sqrt{(s_{11}-s_{22})^2+(s_{22}-s_{33})^2+(s_{33}-s_{11})^2+6(s_{12}^2+s_{23}^2+s_{31}^2)}
\end{equation*}
%@END

%@DOC:SymTensor2::productByRxRT(Tensor2)
%Warning :
%This area is an automatic documentation generated from the DynELA source code.
%Do not change anything in this latex file between this position and the @END keyword.
\textcolor{purple}{\textbf{SymTensor2::productByRxRT(Tensor2)}}\label{SymTensor2::productByRxRT(Tensor2)} : Special combination for a multiplication of 3 second order tensors.\index[DL]{SymTensor2!productByRxRT(Tensor2)}\\ \hspace*{5mm}$\hookrightarrow$ Tensor2

This method defines the product of a symmetric tensor by two rotations defined by the following equation:
\begin{equation*}
\R = \Q \cdot \A \cdot \Q^T =\left[\begin{array}{ccc}
R_0&R_1&R_2\\
&R_3&R_4\\
&&R_5
\end{array}\right]
\end{equation*}
with:
\begin{align*}
R_0&=A_0 Q_0^2 + 2 (A_1 Q_1 + A_2 Q_2)Q_0 + A_3 Q_1^2 + 2 A_4 Q_1 Q_2 + A_5 Q_2^2\\
R_1&=(A_0 Q_3 + A_1 Q_4 + A_2 Q_5)Q_0 + (A_1 Q_3 + A_3 Q_4 + A_4 Q_5)Q_1 + (A_2 Q_3 + A_4 Q_4 + A_5 Q_5)Q_2\\
R_2&= (A_0 Q_6 + A_1 Q_7 + A_2 Q_8)Q_0 + (A_1 Q_6 + A_3 Q_7 + A_4 Q_8)Q_1 + (A_2 Q_6 + A_4 Q_7 + A_5 Q_8)Q_2\\
R_3&=A_0 Q_3^2 + 2 (A_1 Q_4 + A_2 Q_5)Q_3+ A_3 Q_4^2 + 2 A_4 Q_4 Q_5 + A_5 Q_5^2\\
R_4&= (A_0 Q_6 + A_1 Q_7 + A_2 Q_8)Q_3 + (A_1 Q_6 + A_3 Q_7 + A_4 Q_8)Q_4 + (A_2 Q_6 + A_4 Q_7 + A_5 Q_8)Q_5\\
R_5&=A_0 Q_6^2 + 2 (A_1 Q_7 + A_2 Q_8)Q_6+ A_3 Q_7^2 + 2 A_4 Q_7 Q_8 + A_5 Q_8^2
\end{align*}
where $\A$ is a symmetric second order tensor and $\Q$ an orthogonal tensor.
%@END

%@DOC:SymTensor2::productByRTxR(Tensor2)
%Warning :
%This area is an automatic documentation generated from the DynELA source code.
%Do not change anything in this latex file between this position and the @END keyword.
\textcolor{purple}{\textbf{SymTensor2::productByRTxR(Tensor2)}}\label{SymTensor2::productByRTxR(Tensor2)} : Special combination for a multiplication of 3 second order tensors.\index[DL]{SymTensor2!productByRTxR(Tensor2)}\\ \hspace*{5mm}$\hookrightarrow$ Tensor2

This method defines the product of a symmetric tensor by two rotations defined by the following equation:
\begin{equation*}
\R = \Q^T \cdot \A \cdot \Q =\left[\begin{array}{ccc}
R_0&R_1&R_2\\
&R_3&R_4\\
&&R_5
\end{array}\right]
\end{equation*}
with:
\begin{align*}
R_0 &= A_0 Q_0^2 + 2 (A_1 Q_3 + A_2 Q_6) Q_0 + A_3 Q_3^2 + 2 A_4 Q_3 Q_6 + A_5 Q_6^2 \\
R_1 &= A_1 Q_1 Q_3 +A_2 Q_1 Q_6 + (A_0 Q_1 + A_1 Q_4 + A_2 Q_7) Q_0+ A_3 Q_3 Q_4 + A_4 Q_4 Q_6 + A_4 Q_3 Q_7 + A_5 Q_6 Q_7\\
R_2 &= A_1 Q_2 Q_3 + A_2 Q_2 Q_6 + Q_0 (A_0 Q_2 + A_1 Q_5 + A_2 Q_8) + A_3 Q_3 Q_5 + A_4 Q_5 Q_6 + A_4 Q_3 Q_8 + A_5 Q_6 Q_8\\
R_3 &= A_0 Q_1^2 + 2 (A_1 Q_4 + A_2 Q_7)Q_1 + A_3 Q_4^2 + 2 A_4 Q_4 Q_7 + A_5 Q_7^2\\
R_4 &= A_1 Q_2 Q_4 + A_2 Q_2 Q_7 + (A_0 Q_2 + A_1 Q_5 + A_2 Q_8)Q_1 + A_3 Q_4 Q_5 + A_4 Q_5 Q_7 + A_4 Q_4 Q_8 + A_5 Q_7 Q_8\\
R_5 &= A_0 Q_2^2 + 2 (A_1 Q_5 + A_2 Q_8) Q_2 + A_3 Q_5^2 + 2 A_4 Q_5 Q_8 + A_5 Q_8^2
\end{align*}
where $\A$ is a symmetric second order tensor and $\Q$ an orthogonal tensor.
%@END

\section{The Tensor3 library}

The Tensor3 library is used to store third order tensors defined in the \DynELA.

%@DOC:Tensor3::Tensor3()
%Warning :
%This area is an automatic documentation generated from the DynELA source code.
%Do not change anything in this latex file between this position and the @END keyword.
\textcolor{purple}{\textbf{Tensor3::Tensor3(~)}}\label{Tensor3::Tensor3()} : This method is the default constructor of a third order tensor.\index[DL]{Tensor3!Tensor3(~)}\\ \hspace*{5mm}$\hookrightarrow$ Tensor3

All components are initialized to zero by default.
%@END

%@DOC:Tensor3::~Tensor3()
%Warning :
%This area is an automatic documentation generated from the DynELA source code.
%Do not change anything in this latex file between this position and the @END keyword.
\textcolor{purple}{\textbf{Tensor3::$\sim$Tensor3(~)}}\label{Tensor3::~Tensor3()} : Destructor of the Tensor3 class.\index[DL]{Tensor3!$\sim$Tensor3(~)}

%@END

%@DOC:Tensor3::setToUnity()
%Warning :
%This area is an automatic documentation generated from the DynELA source code.
%Do not change anything in this latex file between this position and the @END keyword.
\textcolor{purple}{\textbf{Tensor3::setToUnity(~)}}\label{Tensor3::setToUnity()} : Returns an identity third order tensor.\index[DL]{Tensor3!setToUnity(~)}

\hspace*{10mm}\textcolor{red}{\textbf{Warning :} This method modifies its own argument}

This method transforms the current tensor to a unity tensor.
%@END

\section{The Tensor4 library}

The Tensor4 library is used to store fourth order tensors defined in the \DynELA.

%@DOC:Tensor4::Tensor4()
%Warning :
%This area is an automatic documentation generated from the DynELA source code.
%Do not change anything in this latex file between this position and the @END keyword.
\textcolor{purple}{\textbf{Tensor4::Tensor4(~)}}\label{Tensor4::Tensor4()} : Default constructor of the TenTensor4sor2 class.\index[DL]{Tensor4!Tensor4(~)}\\ \hspace*{5mm}$\hookrightarrow$ Tensor4

All components are initialized to zero by default.
%@END

%@DOC:Tensor4::~Tensor4()
%Warning :
%This area is an automatic documentation generated from the DynELA source code.
%Do not change anything in this latex file between this position and the @END keyword.
\textcolor{purple}{\textbf{Tensor4::$\sim$Tensor4(~)}}\label{Tensor4::~Tensor4()} : Destructor of the Tensor4 class.\index[DL]{Tensor4!$\sim$Tensor4(~)}

%@END

%@DOC:Tensor4::setToUnity()
%Warning :
%This area is an automatic documentation generated from the DynELA source code.
%Do not change anything in this latex file between this position and the @END keyword.
\textcolor{purple}{\textbf{Tensor4::setToUnity(~)}}\label{Tensor4::setToUnity()} : Unity tensor.\index[DL]{Tensor4!setToUnity(~)}

\hspace*{10mm}\textcolor{red}{\textbf{Warning :} This method modifies its own argument}

This method transforms the current tensor to a unity tensor.
%@END
